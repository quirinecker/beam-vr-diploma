Mit BeamVR wurde eine einfache Augmented Virtuality Applikation entwickelt.
Hiermit wurde gezeigt, dass reale Elemente in einer virtuellen welt die Immersion und den realismus stark verstärken.

Durch einen einfachen Balken, welcher sich für gewöhnlich nicht bewegt, wurde eine Augmented Virtuality Applikation entwickelt.
Mit zusätzlichen Funktionalitäten, wie dem Full Body Tracking und einer vielfältigen Umgebung wird gezeigt, dass es mit dieser Kombination möglich ist eine starke Immersion zu erzeugen.

Um eine Applikation wie diese in Betrieb zu nehmen, sind viele Schritte involviert.
Dabei versucht unsere Applikation die Konfigurationen und Schritte zu minimieren.
Beispielsweise benützen wie möglichst viele Informationen von SteamVR.

Für einen gewissen Spannungsaufbau kann die Benutzerin oder der Benutzer auch von dem Balken herunterfallen.
Fällt die Benutzerin oder der Benutzer von dem Balken, fällt dieser ebenso von dem Hochhaus in der Applikation herunter.

Durch viele Einflussfaktoren bei dem Setup der Applikation, kommt es teilweise zu unerwarteten Resultaten.

Beispielsweise kommt es oft zu Drehungen des VR Raumes in der virtuellen Realität.
Durch die Drehung muss der VR Raum neu kalibriert werden oder die physische Anordnung abgeändert werden, da der Balken beispielsweise in die falsche Richtung schaut.
Dies kann bei Veranstaltungen zu minimalen Problemen führen.

Außerdem ist die Stabilität der Software noch nicht auf dem erhofften Stand.
Dadurch können plötzliche Probleme bei dem Full-Body-Tracking und der Balken Kalibrierung auftauchen.

Zukünftig soll die Stabilität der Software noch verbessert werden, damit die Applikation mit einer niedrigeren Fehlerwahrscheinlichkeit präsentierbar ist.

Außerdem würde eine Auswahl von Charakteren und weiteren Karten mehr Abwechslung in die Applikation bringen.

In dem Augmented Virtuality Spektrum besteht noch sehr viel Potenzial.
Dabei besteht die Möglichkeit noch mehr Elemente in die virtuelle Realität einzubauen und damit neue Prinzipien zu entwickeln.

Für nachfolgende Diplomarbeiten könnte, zum Beispiel ein anderes Szenario, gewählt werden.
Hierbei besteht bereits die Idee für einen Balken, welcher an der Spitze eines Zuges angebracht ist.
