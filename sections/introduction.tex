\section{Ausgangssituation}\label{sec: initial_situation}

Es gibt bereits viele VR Applikation, aber was ist genau eine VR Applikation.
Es gibt viele Definitionen für VR ausgeschrieben Virtual Reality.
In dieser Arbeit sprechen wir über die Wirklichkeit welche durch ein head mounted display, oder umgangssprachlich auch eine VR-Brille angezeigt wird.
Somit bezeichnet eine VR Application eine Software, welche in der VR-Brille läuft.

Von diesen VR Applikationen gibt es schon einige.
Die meisten werden im Bereich Videospiele verwendet nach einer Statistik aus Deutschland im Jahre 2021, bei welcher Personen mit einer VR-Brille oder head mounted display befragt worden sind.
Siehe Abbildung~\ref{fig:statistic_usage_vr}~\cite{BITKOM_2021}

\begin{figure}
    \includegraphics[scale=0.5]{pics/statistic_usage_vr}
    \caption{Anwendungsgebiete VR}
    \label{fig:statistic_usage_vr}
\end{figure}

VR Applikationen der Vergangenheit haben sich hauptsächlich auf das Tracking des Kopfes konzentriert.
Das Tracken weiterer Körperteile oder ein sogenanntes Full Body Tracking, also das tracken der Füße, Hände, Hüfte und Kopf war bis vor kurzem ein technisches und vor allem auch ein finanzielles Problem (siehe~\cite{PAVEL_NUZHDIN_2020}).

Im Falle der VR Chat werden zwei Vive Tracker (Siehe~\ref{sec:vive-tracker}) verwendet um den linken und den rechten Fuß zu tracken und ein weiterer um die Hüfte zu tracken.
Für die Hände und den Kopf werden jeweils die Controller (Siehe Abschnitt~\ref{sec:vr-controller})und das VR Headset (Siehe Abschnitt~\ref{sec:vr-headset}) verwebtet.
Somit kann der ganze Körper getrackt werden, da die restlichen positionen berechnetet werden können.
Diese Implementierung kann zum Beispiel bei der Applikation VR Chat gefunden werden (Siehe~\cite{VRCHAT_2021} oben).
In der Dokumentation wird es auch 6PT oder six-point genannt.
VR Chat bietet auch ein weniger genaues Tracking an welches nur einen Tracker, zwei Controller und das VR Headset verwendet.

Ein weiterer Schritt im Reality-Virtual Continuum (siehe~\cite{MILGRAM_1994}) ist die Einbindung realer Gegenstände in die virtuelle Welt.
Man spricht in diesem Fall von Augmented Virtuality.
Dadurch wird der Grad an Immersion (Siehe~\cite{EMEST_ADAMS_2004} und~\cite{BJOERK_2003}), also wie "real" die virtuelle Welt vom Benutzer wahrgenommen wird, erhöht.

\section{Zielsetzung}\label{sec: objective}

Das Ziel dieser Arbeit is eine Applikation der Augmented Virtuality.
Im Fall von Beam VR wird ein Holzsparren, welcher in der realen Welt auf dem Boden liegt, in der virtuellen Welt als Balken, der aus einem Wolkenkratzer ragt, wahrgenommen.
% TODO: Siehe eine Abbildung von der realen Umgebung und eine von der virtuellen Umgebung
Die Applikation soll einem das Gefühl vermitteln, dass man wirklich auf einem Balken steht, welcher von einem Hochhaus wegsteht.
Der Nutzer soll eine Erhöhung spüren sobald er in der virtuellen Rea



Das Ziel von Beam VR ist die physische Realität und die virtuelle Realität zu kombinieren.
Im Falle Beam VR benützen wir hier einen Balken, welcher von einem Hochhaus wegsteht.
Dieser Balken soll in der physischen Realität und in der virtuellen Realität existieren und die gleiche Position einehmen.
Dadurch soll die Immersion entstehen, dass man wirklich auf einem Balken steht, welcher von einem Hochhaus wegsteht.
Auch, wenn man nur in seinem Wohnzimmer oder auch woanders ohne Gefahren auf einem Balken steht.
Beam VR ist nicht der Erfinder dieses Konzeptes.
Mehr dazu gibt es in der kommenden Umfeldanalyse.

