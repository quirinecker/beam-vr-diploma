\section{Hardware}
\lipsum[5-12]

\subsection{VR Headset}
\lipsum[5-12]

\subsection{VR Controller}
\lipsum[5-12]

\subsection{Tracker}
\lipsum[5-12]

\subsection{Lighthouse}
\lipsum[5-12]

\subsection{Wireless Adapter }
\lipsum[5-12]

\section{Software}

\subsection{Game Engine}

Es gibt mehrere Games Engines mit welchen eine VR Applikation geschrieben werden kann.
Folgend ist ein Graph für den Market-share verschiedener Game Engines.
Diese Daten sind aber mit vorsicht zu genießen, da das Skript welche diese Daten geliefert hat nach einigen Kriterien handelt.
Siehe \href{https://www.reddit.com/r/gamedev/comments/8s20qp/i_researched_the_market_share_of_game_engines_on/}{hier}.

\includegraphics[scale=0.5]{pics/game_engine_marketshare}

\subsubsection{Unity}

Unity ist eine Game Engine welche erstmals eine Apple exklusive Game Engine war und von Unity Technologies entwickelt worden ist.
Die Engine wurde weiter entwickelt und kann heute auch auf Windows und auf der Linux plattform benützt werden.
Die Engine ist gratis und wird von vielen als Einsteiger Engine beschrieben.
Auch sie eine Einsteiger Engine genannt wird heißt das nicht, dass sie in keinen professionellen Bereich benützt wird.
Viele bekannte Spiele wurden mit der Unity Engine entwickelt.
Im Besonderen sind viele Handyspiele mit dieser Engine entwickelt.
Spiele wie Pokemon GO, Among us und Hearthstone wurden in der Unity Engine entwickelt.

Vorteile:

\begin{itemize}
    \item Gratis Lizenz für persönlichen Nutzen und für Unternehmen mit unter 100000\$ Einkommen
    \item Programmierbar in der C# Programmiersprache
    \item Es kann fuer alle moeglichen Plattformen ein Programm geschrieben werden
    \begin{itemize}
        \item IOS
        \item Android
        \item Windows
        \item Linux
        \item usw.
    \end{itemize}
    \item verfügbaren Asset-store mit vielen verschiedenen fertigen Assets
\end{itemize}

Nachteile:

\begin{itemize}
    \item weniger Market-share (siehe Kreisdiagramm von oben)
    \item geschlossener Source Code
\end{itemize}

\subsection{VR Plugin}
\lipsum[5-12]

\subsection{Steam}
\lipsum[5-12]

\subsection{Vive Wireless}
\lipsum[5-12]

\subsection{Final IK Plugin}
\lipsum[5-12]

\subsection{IDE}
\lipsum[5-12]

\subsection{Modellierung}
\lipsum[5-12]
